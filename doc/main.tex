%
% File acl2014.tex
%
% Contact: koller@ling.uni-potsdam.de, yusuke@nii.ac.jp
%%
%% Based on the style files for ACL-2013, which were, in turn,
%% Based on the style files for ACL-2012, which were, in turn,
%% based on the style files for ACL-2011, which were, in turn, 
%% based on the style files for ACL-2010, which were, in turn, 
%% based on the style files for ACL-IJCNLP-2009, which were, in turn,
%% based on the style files for EACL-2009 and IJCNLP-2008...

%% Based on the style files for EACL 2006 by 
%%e.agirre@ehu.es or Sergi.Balari@uab.es
%% and that of ACL 08 by Joakim Nivre and Noah Smith

\documentclass[11pt]{article}
\usepackage{acl2014}
\usepackage{times}
\usepackage{url}
\usepackage{latexsym}

%\setlength\titlebox{5cm}

% You can expand the titlebox if you need extra space
% to show all the authors. Please do not make the titlebox
% smaller than 5cm (the original size); we will check this
% in the camera-ready version and ask you to change it back.


\title{Ling 573 Summarization Presentation}

\author{Thomas Marsh \\
  University of Washington \\
  ling 573, Spring 2015 \\
  {\tt sugarork@uw.edu\hspace{5mm}} \\\And
  Michael Roylance \\
  University of Washington \\
  ling 573, Spring 2015 \\
  {\tt roylance@uw.edu} \\
\\\And
  Brandon Gahler  \\
  University of Washington \\
  ling 573, Spring 2015 \\
  {\tt bjg6@uw.edu} \\} 
  

\date{Spring 2015}
\begin{document}
\maketitle
\begin{abstract}
  Abstract is a strange name for a summary.  It is not really abstract at all.
\end{abstract}


\section{Introduction}

This is the introduction.  It will introduce things. 

\section{System Overview}

A description of the major design, methodological, and algorithmic decisions in your project. It often includes a schematic of the system architecture.

\section{Approach}

This section should provide the details of the major subcomponents of your system.

\section{Results}

This section should present the major results of the formal evaluation of your system and components.

\section{Discussion}

This section will analyze your results in a bit more detail. It is an appropriate location for error analysis and assessment of the strengths and weaknesses of the different components.

\section{Conclusion}
Conclusions have been made as can be seen from the following Nenkova, Radev, and Jones \cite{nenkova2007pyramid} \cite{SpärckJones20071449} \cite{radev2001experiments}

\bibliographystyle{acl}
\bibliography{bibReferences}

\end{document}