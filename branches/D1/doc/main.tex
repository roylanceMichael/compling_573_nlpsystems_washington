%
% File acl2014.tex
%
% Contact: koller@ling.uni-potsdam.de, yusuke@nii.ac.jp
%%
%% Based on the style files for ACL-2013, which were, in turn,
%% Based on the style files for ACL-2012, which were, in turn,
%% based on the style files for ACL-2011, which were, in turn, 
%% based on the style files for ACL-2010, which were, in turn, 
%% based on the style files for ACL-IJCNLP-2009, which were, in turn,
%% based on the style files for EACL-2009 and IJCNLP-2008...

%% Based on the style files for EACL 2006 by 
%%e.agirre@ehu.es or Sergi.Balari@uab.es
%% and that of ACL 08 by Joakim Nivre and Noah Smith

\documentclass[11pt]{article}
\usepackage{acl2014}
\usepackage{times}
\usepackage{url}
\usepackage{latexsym}

%\setlength\titlebox{5cm}

% You can expand the titlebox if you need extra space
% to show all the authors. Please do not make the titlebox
% smaller than 5cm (the original size); we will check this
% in the camera-ready version and ask you to change it back.


\title{Ling 573 Summarization Presentation}

\author{Thomas Marsh \\
  University of Washington \\
  ling 573, Spring 2015 \\
  {\tt pretend.intellectual@gmail.com\hspace{5mm}} \\\And
  Michael Roylance \\
  University of Washington \\
  ling 573, Spring 2015 \\
  {\tt roylance.michael@gmail.com} \\}

\date{Spring 2015}

\begin{document}
\maketitle
\begin{abstract}
  Abstract is a strange name for a summary.  It is not really abstract at all.
\end{abstract}


\section{Introduction}

This is the introduction.  It will introduce things. 

\section{System Overview}

A description of the major design, methodological, and algorithmic decisions in your project. It often includes a schematic of the system architecture.

\section{Approach}

This section should provide the details of the major subcomponents of your system.

\section{Results}

This section should present the major results of the formal evaluation of your system and components.

\section{Discussion}

This section will analyze your results in a bit more detail. It is an appropriate location for error analysis and assessment of the strengths and weaknesses of the different components.

\section{Conclusion}

Conclusions can be drawn.

% include your own bib file like this:
%\bibliographystyle{acl}
%\bibliography{acl2014}

\begin{thebibliography}{}

\bibitem[\protect\citename{Aho and Ullman}1972]{Aho:72}
Alfred~V. Aho and Jeffrey~D. Ullman.
\newblock 1972.
\newblock {\em The Theory of Parsing, Translation and Compiling}, volume~1.
\newblock Prentice-{Hall}, Englewood Cliffs, NJ.

\bibitem[\protect\citename{{American Psychological Association}}1983]{APA:83}
{American Psychological Association}.
\newblock 1983.
\newblock {\em Publications Manual}.
\newblock American Psychological Association, Washington, DC.

\bibitem[\protect\citename{{Association for Computing Machinery}}1983]{ACM:83}
{Association for Computing Machinery}.
\newblock 1983.
\newblock {\em Computing Reviews}, 24(11):503--512.

\bibitem[\protect\citename{Chandra \bgroup et al.\egroup }1981]{Chandra:81}
Ashok~K. Chandra, Dexter~C. Kozen, and Larry~J. Stockmeyer.
\newblock 1981.
\newblock Alternation.
\newblock {\em Journal of the Association for Computing Machinery},
  28(1):114--133.

\bibitem[\protect\citename{Gusfield}1997]{Gusfield:97}
Dan Gusfield.
\newblock 1997.
\newblock {\em Algorithms on Strings, Trees and Sequences}.
\newblock Cambridge University Press, Cambridge, UK.

\end{thebibliography}

\end{document}
